
\section{MARS方法\cite{zhang2016mars}}

在界面追踪(IT)问题当中,我们通常得到一个先验的速度场
$\mathbf{u}(\mathbf{x},t),$
每个流体相通过这个速度场推进,对应的常微分方程如下:
\begin{equation}
\label{volEquation}
\frac{\mathrm{d}\, \mathbf{x}}{\mathrm{d}\, t}=\mathbf{u}(\mathbf{x},t),
\end{equation}
若速度场$\mathbf{u}(\mathbf{x},t)$对时间$t$连续,
在空间$\mathbb{R}^{\textup{D}}$上Lipschitz连续,
则其存在唯一解。由解的唯一性可得流映射 $\phi: \mathbb{R}^{\textup{D}}\times \mathbb{R}\times\mathbb{R}\rightarrow \mathbb{R}^{\text{D}}$.
\begin{equation} 
\label{eq:tracing}
\renewcommand{\arraystretch}{1.3}
\left\{
\begin{array}{l}
\phi_{t_0}^{+k}(p) := p(t_0+k) 
= p(t_0) + \int_{t_0}^{t_0+k} \mathbf{u}(p(t),t)\,\mathrm{d}\, t,
\\
\phi_{t_0}^{-k}(p) := p(t_0-k) 
= p(t_0) + \int_{t_0}^{t_0-k} \mathbf{u}(p(t),t)\,\mathrm{d}\,t.
\end{array}
\right.
\end{equation}
将流映射推广到任意点集有:
\begin{displaymath}
\phi_{t_0}^{\pm k}({\mathcal M})  
= \{\phi_{t_0}^{\pm k}(p) : p\in {\mathcal M}\}.
\end{displaymath}
% 记 $\overleftarrow{M}:=\phi_{t_0+k}^{-k}(M)$, $\overrightarrow{M}:=\phi_{t_0}^{+k}(M)$, 
% 其中$t_0$表示当前时刻,$k$表示时间步间隔。

\begin{defn}
	一个半离散流映射$\mathring{ \phi} : \mathbb{S}^{\textup{D}}\rightarrow \mathbb{S}^{\textup{D}}$
	是用一个一致的时间积分方法求解\eqref{volEquation}式得到的精确流映射$\phi$的近似.
	称$\mathring{ \phi}_{t_0}^{nk}$在时间上是$\kappa$阶精度的,
	若对于任意的$p(t_0)\in \mathbb{R}^{\textup{D}}$, $t_0+nk\leq T$,
	有$\mathring{ \phi}_{t_0}^{nk}=\phi_{t_0}^{nk}(p)+O(k^{\kappa})$.
\end{defn}
\begin{defn}
	一个离散流映射$\varphi : \mathbb{S}_q^{\textup{D}}\rightarrow\mathbb{S}_q^{\textup{D}}$
	是半离散流映射$\mathring{ \phi}$的近似,由以下两步所得,
	\begin{enumerate}
		\item 把$\mathring{ \phi}$作用到边界$\mathcal{P}$的顶点($0$-单形)上;
		\item 构造一个$\mathring{ \phi}(\partial\mathcal{P})$的多项式与$\partial \mathcal{P}$同胚。
		当$q=1$时,在二维情况下,边界由相连的直线近似;
		在三维情况下,边界由多个三角平面组成的多面体来近似。
	\end{enumerate}
\end{defn}

\begin{defn}
	\label{defn:MARS}
	MARS方法是具有以下形式的界面追踪方法
	\begin{equation*}
	\label{eq:interfaceTrackingMethod}
	{\mathcal M}^{n+1}={\mathfrak L}_{\mathrm{Mars}}^n {\mathcal M}^n
	:= \left(\chi_{n+1}\circ\varphi_{t_n}^k\circ\psi_n
	\right){\mathcal M}^n,
	\end{equation*}
	其中$\mathcal{M}^n\in\mathbb{Y}$ 是$\mathcal{M}(t_n)\in\mathbb{Y}$的近似,
	$\varphi:\mathbb{Y}\rightarrow
	\mathbb{Y}$ 是\eqref{eq:tracing}式中$\phi$的离散流映射, 
	$\psi_n:\mathbb{Y}\rightarrow \mathbb{Y}$是预处理算子,
	$\chi_{n+1}:\mathbb{Y}\rightarrow \mathbb{Y}$是后处理算子。
\end{defn}

\begin{defn}
  考虑一种MARS方法,
  其中映射操作$\varphi$根据精确的映射$\phi$
  在时间上离散化后得到$\mathring{ \phi}$,
  再在空间上离散$\mathring{ \phi}$得到。
  MARS方法在$t_n=t_0+nk$时刻的
  时间积分误差$E^{\mathrm{ODE}}(t_n),$
  表示误差$E^{\mathrm{REP}}(t_n),$
  扩充误差$E^{\mathrm{AUG}}(t_n),$
  映射误差$E^{\mathrm{MAP}}(t_n),$
  以及调整误差$E^{\mathrm{ADJ}}(t_n)$分别如下:
  \begin{equation}\label{defn:error}
    \left\{
      \begin{array}{rl}
	E^{\mathrm{ODE}}(t_n) &:= \| \phi_{t_0}^{nk}(\mathcal{M}^0) \oplus \mathring{ \phi}_{t_0}^{nk}(\mathcal{M}^0) \|;\\[0.2cm]
	E^{\mathrm{REP}}(t_n) &:= \| \phi_{t_0}^{nk}(\mathcal{M}(t_0)) \oplus  \phi_{t_0}^{nk}(\mathcal{M}^0) \|;\\[0.2cm]
	\varepsilon_i^{\mathrm{AUG}} &:=  ( \psi_i \mathcal{M}^i )\oplus \mathcal{M}^i, \\[0.2cm]
	E^{\mathrm{AUG}}(t_n) &:= \| \oplus_{j=0}^{n-1} \mathring{ \phi}_{t_j}^{(n-j)k}\varepsilon_j^{\mathrm{AUG}} \|;\\[0.2cm]
	\varepsilon_i^{\mathrm{MAP}}(t_n)& := \mathring{ \phi}_{t_i}^k(\psi_i\mathcal{M}^i)\oplus \varphi_{t_i}^k(\psi_i\mathcal{M}^i),\\[0.2cm]
	E^{\mathrm{MAP}}(t_n) &:=  \| \oplus_{j=1}^n \mathring{ \phi}_{t_j}^{(n-j)k}\varepsilon_{j-1}^{\mathrm{MAP}} \|;\\[0.2cm]
	\varepsilon_{i+1}^{\mathrm{ADJ}} &:= (\varphi_{t_i}^k\psi_i\mathcal{M}^i)\oplus\mathcal{M}^{i+1},\\[0.2cm]
	E^{\mathrm{ADJ}}(t_n) &:= \| \oplus_{j=1}^n\mathring{ \phi}_{t_j}^{(n-j)k}\varepsilon_j^{\mathrm{ADJ}}\|.\\[0.2cm]
      \end{array}\right.
  \end{equation}
  $\varepsilon_i^{\mathrm{AUG}}$, $\varepsilon_i^{\mathrm{MAP}}$, $\varepsilon_{i+1}^{\mathrm{ADJ}}$称为第$i$个时间步的误差区域。
\end{defn}
这里$E^{\mathrm{ODE}}$是用半离散流映射$\mathring{ \phi}$近似精确流映射$\phi$造成的误差,
$E^{\mathrm{REP}}$是在初始时刻用半代数集表示流相的误差,
$E^{\mathrm{AUG}}$是扩充半代数集的累计误差,
$E^{\mathrm{MAP}}$是用映射操作近似半离散流映射的累计误差,
$E^{\mathrm{ADJ}}$是调整半代数集的像的累计误差。

\begin{thm}
定义\ref{defn:MARS}中的MARS方法是殷空间上的三个算子的复合,
通过\eqref{defn:error}式定义的各类单一误差以及对每个算子的分析,
得MARS方法的整体误差满足:
 \begin{equation}
\label{eq:E1parts}
E_1(t_n)  \le E^{\mathrm{ODE}} + E^{\mathrm{MAP}}
+ E^{\mathrm{REP}} + E^{\mathrm{AUG}} + E^{\mathrm{ADJ}}.
\end{equation}
这里的 $E^{\mathrm{ODE}},$$E^{\mathrm{MAP}},$
$E^{\mathrm{REP}},$ $E^{\mathrm{AUG}},$$E^{\mathrm{ADJ}}$
均由\eqref{defn:error}给出。
\end{thm}

\begin{pro}
  见\cite{zhang2016mars}. 
\end{pro}

\begin{prop}
  \label{ODEerror}
  如果一个MARS方法的半离散流映射$\mathring{ \phi}$有$\kappa$阶精度,
  则对任意$t_n = t_0 + nk(k = O(\frac{1}{n}))$有
  \begin{displaymath}
    E^{\rm{ODE}}(t_n) = O(k^{\kappa}).
  \end{displaymath}
\end{prop}
\begin{pro}
  见\cite{zhang2016mars}.
\end{pro}
