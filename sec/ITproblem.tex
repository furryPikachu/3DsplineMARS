
\section{界面追踪问题}
利用殷空间\cite{zhang2018fourth}的概念描述界面追踪问题。
\begin{defn}[界面追踪问题]
  \label{IT}
  用一个殷集$\mathcal{M}(t) \in \mathbb{Y}$表示被追踪物质$M$在$t$时刻所处区域,
  由初始条件$\mathcal{M}(t_0)$和给定流速场$\boldsymbol{u}(\boldsymbol{x},t)(t\in [t_0,T])$
  确定$\mathcal{M}(T)$的问题称为界面追踪问题。
\end{defn}
为了度量界面追踪的精度,利用\cite{zhang2018fourth}中度量(2.10)定义误差
\begin{equation}
  \label{eq:geomError}
  E_1(t_n) := \|\mathcal{M}(t_n) \oplus
  \mathcal{M}^{n}\| = \sum_{{\mathcal C}\subset \Omega}
  \left\|\mathcal{M}_{\mathcal C}(t_n) \oplus
    \mathcal{M}_{\mathcal C}^{n}\right\|,
\end{equation}
其中,$\mathcal{M}(t_n)$表示$t_n$时刻精确流体区域,
$\mathcal{M}^{n}$是$\mathcal{M}(t_n)$的一个近似,
所有控制体$\mathcal{C}$的集合形成计算域$\Omega$的一个划分。

\begin{defn}[界面追踪方法的精度]
  数值求解界面追踪问题(定义\ref{IT}),称一个界面追踪方法$\mathcal{L}_{IT}$
  \begin{equation}
    \label{eq:ITM}
    \mathcal{M}^{n+1}=\mathcal{L}_{IT}(\mathcal{M}(t_n),\boldsymbol{u})\approx \mathcal{M}(t_{n+1}),
  \end{equation}
  在$1-$范数下有$\beta$阶精度,若
  \begin{displaymath}
    E_1(T) = O(k^{\beta})(k \to 0).
  \end{displaymath}
  称$\mathcal{L}_{IT}$是一致的,若$\beta > 0$.
\end{defn}

