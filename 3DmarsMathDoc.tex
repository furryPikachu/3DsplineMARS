\documentclass[11pt]{article}

\usepackage{geometry}
% make full use of A4 papers
\geometry{margin=1.5cm, vmargin={0pt,1cm}}
\setlength{\topmargin}{-1cm}
\setlength{\paperheight}{29.7cm}
\setlength{\textheight}{25.1cm}
\usepackage[BoldFont,SlantFont,CJKchecksingle]{xeCJK}%CJKsetspaces
\setCJKmainfont[BoldFont=SimHei]{SimSun}
\setCJKmonofont{SimSun}% 设置缺省中文字体
\parindent 2em   %段首缩进

\usepackage{setspace}
% \setlength{\baselineskip}{25pt}
\linespread{1.25}
% auto adjust the marginals
\usepackage{marginfix}
\usepackage{amsfonts}
\usepackage{amsmath}
\usepackage{amssymb}
\usepackage{amsthm}
\usepackage{CJKutf8}   % for Chinese characters
\usepackage{enumerate}
\usepackage{fancyhdr}
\usepackage{graphicx}  % for figures
\usepackage{layout}
\usepackage{mathrsfs}  
\usepackage{multicol}  % multiple columns to reduce number of pages
%\usepackage{natbib}
\usepackage{subfigure}
\usepackage{tcolorbox}
\usepackage{tikz-cd}
\usepackage{extarrows}
\usepackage{float}

\usepackage[backend=biber,style=gb7714-2015]{biblatex}
\addbibresource[location=local]{bib/3Dmars.bib}

%------------------
% common commands %
%------------------
% differentiation
\newcommand{\aff}{\mathrm{aff}}
\newcommand{\conv}{\mathrm{conv}}
\newcommand{\VertSet}{\mathrm{Vert}}
\newcommand{\Star}{\mathrm{st}}
\newcommand{\gen}[1]{\left\langle #1 \right\rangle}
\newcommand{\dif}{\mathrm{d}}
\newcommand{\difPx}[1]{\frac{\partial #1}{\partial x}}
\newcommand{\difPy}[1]{\frac{\partial #1}{\partial y}}
\newcommand{\Dim}{\mathrm{D}}
\newcommand{\avg}[1]{\left\langle #1 \right\rangle}
\newcommand{\sgn}{\mathrm{sgn}}
\newcommand{\Span}{\mathrm{span}}
\newcommand{\dom}{\mathrm{dom}}
\newcommand{\Arity}{\mathrm{arity}}
\newcommand{\Int}{\mathrm{Int}}
\newcommand{\Ext}{\mathrm{Ext}}
\newcommand{\Cl}{\mathrm{Cl}}
\newcommand{\Fr}{\mathrm{Fr}}
% group is generated by
\newcommand{\grb}[1]{\left\langle #1 \right\rangle}
% rank
\newcommand{\rank}{\mathrm{rank}}
\newcommand{\Iden}{\mathrm{Id}}
\newcommand{\Obj}[1]{\mathrm{obj}\,{\mathscr #1}}
\newcommand{\Hom}{\mathrm{Hom}}

% this environment is for solutions of examples and exercises
\newenvironment{solution}%
{\noindent\textbf{Solution.}}%
{\qedhere}

%the following command is for disabling environments
%  so that their contents do not show up in the pdf.
\makeatletter
\newcommand{\voidenvironment}[1]{%
  \expandafter\providecommand\csname env@#1@save@env\endcsname{}%
  \expandafter\providecommand\csname env@#1@process\endcsname{}%
  \@ifundefined{#1}{}{\RenewEnviron{#1}{}}%
}
\makeatother

%---------------------------------------------
% commands specifically for complex analysis %
%---------------------------------------------
% complex conjugate
\newcommand{\ccg}[1]{\overline{#1}}
% the imaginary unit
\newcommand{\ii}{\mathbf{i}}
%\newcommand{\ii}{\boldsymbol{i}}
% the real part
\newcommand{\Rez}{\mathrm{Re}\,}
% the imaginary part
\newcommand{\Imz}{\mathrm{Im}\,}
% punctured complex plane
\newcommand{\pcp}{\mathbb{C}^{\bullet}}
% the principle branch of the logarithm
\newcommand{\Log}{\mathrm{Log}}
% the principle value of a nonzero complex number
\newcommand{\Arg}{\mathrm{Arg}}
\newcommand{\Null}{\mathrm{null}\,}
\newcommand{\Range}{\mathrm{range}\,}
\newcommand{\Ker}{\mathrm{ker}}
\newcommand{\Iso}{\mathrm{Iso}}
\newcommand{\Aut}{\mathrm{Aut}}
\newcommand{\ord}{\mathrm{ord}}
\newcommand{\Res}{\mathrm{Res}}
%\newcommand{\GL2R}{\mathrm{GL}(2,\mathbb{R})}
\newcommand{\GL}{\mathrm{GL}}
\newcommand{\SL}{\mathrm{SL}}
\newcommand{\Dist}[2]{\left|{#1}-{#2}\right|}



%----------------------------------------
% theorem and theorem-like environments %
%----------------------------------------
% \numberwithin{equation}{chapter}
% \theoremstyle{definition}

% \newtheorem{thm}{Theorem}[chapter]
% \newtheorem{axm}[thm]{Axiom}
% \newtheorem{alg}[thm]{Algorithm}
% \newtheorem{asm}[thm]{Assumption}
% \newtheorem{defn}[thm]{Definition}
% \newtheorem{frm}[thm]{Formula}
% \newtheorem{prop}[thm]{Proposition}
% \newtheorem{qst}[thm]{Question}
% \newtheorem{rul}[thm]{Rule}
% \newtheorem{coro}[thm]{Corollary}
% \newtheorem{lem}[thm]{Lemma}
% \newtheorem{exm}{Example}[chapter]
% \newtheorem{rem}{Remark}[chapter]
% \newtheorem{exc}[exm]{Exercise}
% \newtheorem{ntn}{Notation}


%----------------------
% the end of preamble %
% ----------------------


\begin{document}
% \begin{CJK*}{UTF8}{gkai}
% 		\CJKindent
 % \makeatletter %将文献引用作为上标出现,增加括号,
 % \def\@cite#1#2{\textsuperscript{[{#1\if@tempswa , #2\fi}]}}

 % \makeatother
 \theoremstyle{definition}%amsthm 宏包 标签正体,内容正体
 \newtheorem{thm}{{定理}}
 \newtheorem{lem}[thm]{{引理}}
 \newtheorem{defn}[thm]{{定义}}
\newtheorem{prop}[thm]{{性质}}
 \newtheorem{cor}[thm]{{推论}}
 \newtheorem{nota}[thm]{{记号}}
 \newtheorem{exm}{{例}}
 \newtheorem{rem}[exm]{{注}}
 \newtheorem*{pro}{证明}

\title{三维高阶界面追踪}
\author{应林洁,李祎玲,金鹏炜}
\date{}
\maketitle


\section{界面追踪问题}
利用殷空间\cite{zhang2018fourth}的概念描述界面追踪问题。
\begin{defn}[界面追踪问题]
  \label{IT}
  用一个殷集$\mathcal{M}(t) \in \mathbb{Y}$表示被追踪物质$M$在$t$时刻所处区域,
  由初始条件$\mathcal{M}(t_0)$和给定流速场$\boldsymbol{u}(\boldsymbol{x},t)(t\in [t_0,T])$
  确定$\mathcal{M}(T)$的问题称为界面追踪问题。
\end{defn}
为了度量界面追踪的精度,利用\cite{zhang2018fourth}中度量(2.10)定义误差
\begin{equation}
  \label{eq:geomError}
  E_1(t_n) := \|\mathcal{M}(t_n) \oplus
  \mathcal{M}^{n}\| = \sum_{{\mathcal C}\subset \Omega}
  \left\|\mathcal{M}_{\mathcal C}(t_n) \oplus
    \mathcal{M}_{\mathcal C}^{n}\right\|,
\end{equation}
其中,$\mathcal{M}(t_n)$表示$t_n$时刻精确流体区域,
$\mathcal{M}^{n}$是$\mathcal{M}(t_n)$的一个近似,
所有控制体$\mathcal{C}$的集合形成计算域$\Omega$的一个划分。

\begin{defn}[界面追踪方法的精度]
  数值求解界面追踪问题(定义\ref{IT}),称一个界面追踪方法$\mathcal{L}_{IT}$
  \begin{equation}
    \label{eq:ITM}
    \mathcal{M}^{n+1}=\mathcal{L}_{IT}(\mathcal{M}(t_n),\boldsymbol{u})\approx \mathcal{M}(t_{n+1}),
  \end{equation}
  在$1-$范数下有$\beta$阶精度,若
  \begin{displaymath}
    E_1(T) = O(k^{\beta})(k \to 0).
  \end{displaymath}
  称$\mathcal{L}_{IT}$是一致的,若$\beta > 0$.
\end{defn}



\section{MARS方法\cite{zhang2016mars}}

在界面追踪(IT)问题当中,我们通常得到一个先验的速度场
$\mathbf{u}(\mathbf{x},t),$
每个流体相通过这个速度场推进,对应的常微分方程如下:
\begin{equation}
\label{volEquation}
\frac{\mathrm{d}\, \mathbf{x}}{\mathrm{d}\, t}=\mathbf{u}(\mathbf{x},t),
\end{equation}
若速度场$\mathbf{u}(\mathbf{x},t)$对时间$t$连续,
在空间$\mathbb{R}^{\textup{D}}$上Lipschitz连续,
则其存在唯一解。由解的唯一性可得流映射 $\phi: \mathbb{R}^{\textup{D}}\times \mathbb{R}\times\mathbb{R}\rightarrow \mathbb{R}^{\text{D}}$.
\begin{equation} 
\label{eq:tracing}
\renewcommand{\arraystretch}{1.3}
\left\{
\begin{array}{l}
\phi_{t_0}^{+k}(p) := p(t_0+k) 
= p(t_0) + \int_{t_0}^{t_0+k} \mathbf{u}(p(t),t)\,\mathrm{d}\, t,
\\
\phi_{t_0}^{-k}(p) := p(t_0-k) 
= p(t_0) + \int_{t_0}^{t_0-k} \mathbf{u}(p(t),t)\,\mathrm{d}\,t.
\end{array}
\right.
\end{equation}
将流映射推广到任意点集有:
\begin{displaymath}
\phi_{t_0}^{\pm k}({\mathcal M})  
= \{\phi_{t_0}^{\pm k}(p) : p\in {\mathcal M}\}.
\end{displaymath}
% 记 $\overleftarrow{M}:=\phi_{t_0+k}^{-k}(M)$, $\overrightarrow{M}:=\phi_{t_0}^{+k}(M)$, 
% 其中$t_0$表示当前时刻,$k$表示时间步间隔。

\begin{defn}
	一个半离散流映射$\mathring{ \phi} : \mathbb{S}^{\textup{D}}\rightarrow \mathbb{S}^{\textup{D}}$
	是用一个一致的时间积分方法求解\eqref{volEquation}式得到的精确流映射$\phi$的近似.
	称$\mathring{ \phi}_{t_0}^{nk}$在时间上是$\kappa$阶精度的,
	若对于任意的$p(t_0)\in \mathbb{R}^{\textup{D}}$, $t_0+nk\leq T$,
	有$\mathring{ \phi}_{t_0}^{nk}=\phi_{t_0}^{nk}(p)+O(k^{\kappa})$.
\end{defn}
\begin{defn}
	一个离散流映射$\varphi : \mathbb{S}_q^{\textup{D}}\rightarrow\mathbb{S}_q^{\textup{D}}$
	是半离散流映射$\mathring{ \phi}$的近似,由以下两步所得,
	\begin{enumerate}
		\item 把$\mathring{ \phi}$作用到边界$\mathcal{P}$的顶点($0$-单形)上;
		\item 构造一个$\mathring{ \phi}(\partial\mathcal{P})$的多项式与$\partial \mathcal{P}$同胚。
		当$q=1$时,在二维情况下,边界由相连的直线近似;
		在三维情况下,边界由多个三角平面组成的多面体来近似。
	\end{enumerate}
\end{defn}

\begin{defn}
	\label{defn:MARS}
	MARS方法是具有以下形式的界面追踪方法
	\begin{equation*}
	\label{eq:interfaceTrackingMethod}
	{\mathcal M}^{n+1}={\mathfrak L}_{\mathrm{Mars}}^n {\mathcal M}^n
	:= \left(\chi_{n+1}\circ\varphi_{t_n}^k\circ\psi_n
	\right){\mathcal M}^n,
	\end{equation*}
	其中$\mathcal{M}^n\in\mathbb{Y}$ 是$\mathcal{M}(t_n)\in\mathbb{Y}$的近似,
	$\varphi:\mathbb{Y}\rightarrow
	\mathbb{Y}$ 是\eqref{eq:tracing}式中$\phi$的离散流映射, 
	$\psi_n:\mathbb{Y}\rightarrow \mathbb{Y}$是预处理算子,
	$\chi_{n+1}:\mathbb{Y}\rightarrow \mathbb{Y}$是后处理算子。
\end{defn}

\begin{defn}
  考虑一种MARS方法,
  其中映射操作$\varphi$根据精确的映射$\phi$
  在时间上离散化后得到$\mathring{ \phi}$,
  再在空间上离散$\mathring{ \phi}$得到。
  MARS方法在$t_n=t_0+nk$时刻的
  时间积分误差$E^{\mathrm{ODE}}(t_n),$
  表示误差$E^{\mathrm{REP}}(t_n),$
  扩充误差$E^{\mathrm{AUG}}(t_n),$
  映射误差$E^{\mathrm{MAP}}(t_n),$
  以及调整误差$E^{\mathrm{ADJ}}(t_n)$分别如下:
  \begin{equation}\label{defn:error}
    \left\{
      \begin{array}{rl}
	E^{\mathrm{ODE}}(t_n) &:= \| \phi_{t_0}^{nk}(\mathcal{M}^0) \oplus \mathring{ \phi}_{t_0}^{nk}(\mathcal{M}^0) \|;\\[0.2cm]
	E^{\mathrm{REP}}(t_n) &:= \| \phi_{t_0}^{nk}(\mathcal{M}(t_0)) \oplus  \phi_{t_0}^{nk}(\mathcal{M}^0) \|;\\[0.2cm]
	\varepsilon_i^{\mathrm{AUG}} &:=  ( \psi_i \mathcal{M}^i )\oplus \mathcal{M}^i, \\[0.2cm]
	E^{\mathrm{AUG}}(t_n) &:= \| \oplus_{j=0}^{n-1} \mathring{ \phi}_{t_j}^{(n-j)k}\varepsilon_j^{\mathrm{AUG}} \|;\\[0.2cm]
	\varepsilon_i^{\mathrm{MAP}}(t_n)& := \mathring{ \phi}_{t_i}^k(\psi_i\mathcal{M}^i)\oplus \varphi_{t_i}^k(\psi_i\mathcal{M}^i),\\[0.2cm]
	E^{\mathrm{MAP}}(t_n) &:=  \| \oplus_{j=1}^n \mathring{ \phi}_{t_j}^{(n-j)k}\varepsilon_{j-1}^{\mathrm{MAP}} \|;\\[0.2cm]
	\varepsilon_{i+1}^{\mathrm{ADJ}} &:= (\varphi_{t_i}^k\psi_i\mathcal{M}^i)\oplus\mathcal{M}^{i+1},\\[0.2cm]
	E^{\mathrm{ADJ}}(t_n) &:= \| \oplus_{j=1}^n\mathring{ \phi}_{t_j}^{(n-j)k}\varepsilon_j^{\mathrm{ADJ}}\|.\\[0.2cm]
      \end{array}\right.
  \end{equation}
  $\varepsilon_i^{\mathrm{AUG}}$, $\varepsilon_i^{\mathrm{MAP}}$, $\varepsilon_{i+1}^{\mathrm{ADJ}}$称为第$i$个时间步的误差区域。
\end{defn}
这里$E^{\mathrm{ODE}}$是用半离散流映射$\mathring{ \phi}$近似精确流映射$\phi$造成的误差,
$E^{\mathrm{REP}}$是在初始时刻用半代数集表示流相的误差,
$E^{\mathrm{AUG}}$是扩充半代数集的累计误差,
$E^{\mathrm{MAP}}$是用映射操作近似半离散流映射的累计误差,
$E^{\mathrm{ADJ}}$是调整半代数集的像的累计误差。

\begin{thm}
定义\ref{defn:MARS}中的MARS方法是殷空间上的三个算子的复合,
通过\eqref{defn:error}式定义的各类单一误差以及对每个算子的分析,
得MARS方法的整体误差满足:
 \begin{equation}
\label{eq:E1parts}
E_1(t_n)  \le E^{\mathrm{ODE}} + E^{\mathrm{MAP}}
+ E^{\mathrm{REP}} + E^{\mathrm{AUG}} + E^{\mathrm{ADJ}}.
\end{equation}
这里的 $E^{\mathrm{ODE}},$$E^{\mathrm{MAP}},$
$E^{\mathrm{REP}},$ $E^{\mathrm{AUG}},$$E^{\mathrm{ADJ}}$
均由\eqref{defn:error}给出。
\end{thm}

\begin{pro}
  见\cite{zhang2016mars}. 
\end{pro}

\begin{prop}
  \label{ODEerror}
  如果一个MARS方法的半离散流映射$\mathring{ \phi}$有$\kappa$阶精度,
  则对任意$t_n = t_0 + nk(k = O(\frac{1}{n}))$有
  \begin{displaymath}
    E^{\rm{ODE}}(t_n) = O(k^{\kappa}).
  \end{displaymath}
\end{prop}
\begin{pro}
  见\cite{zhang2016mars}.
\end{pro}

\section{算法}
\label{sec:algorithm}
\input{sec/algorithm.tex}

\section{局部多项式曲面构造}
% \noindent{\parbox{\textwidth}{{\wn{Question:}
%       几阶的多项式曲面可以使得界面追踪精度达到4阶?}}}
% \noindent{\fbox{\parbox{\textwidth}{{\wn{Question:}
%        几阶的多项式曲面可以使得界面追踪精度达到4阶?
%        \uline{u*}}}}}
 % \begin{itemize}
 % \item 推广一元多项式插值余项的推导方式?
 %   ($\mathrm{R}^m\rightarrow \mathrm{R}^n的Roller's Theorem$)
 % \item 参考张量曲面的误差分析
 % \item 转化成点在曲线上分析?
 % \end{itemize}
\begin{rem}
  由于我们用于追踪流体界面的点在空间上是散乱分布的,
  而且已经有三角剖分的结构,因此我们考虑基于三角剖分的二维样条函数。
\end{rem}
\subsection{Bernstein 基多项式}
%%%%%%%%%%%%%%%%%%%%%%%%%%%%% Constants
% a.向前追踪边界上的示踪点b.如果下一时刻的三角性边长大于$h_L$,
% 回溯原来的三角形,在原来的三角形边长中增加中点对应的插值曲面上的点作为新的示踪点

%  插值曲面的构造。

% 将三角网格点投影在x-y平面,向外找star上的示踪点投影到x-y平面。
% 如果一层star没有合适的十个点就向外再裹一层。

% 取最外面的三个点构成包裹的三角形T,
% 运用三角形的重心坐标和Bernstein basis polynomials of degree d relative
% to T,
\begin{nota}
  $d$阶多项式空间
  \begin{displaymath}
{\cal P}_d:=\left\{ p(x,y) = \sum_{0\leq i+j\leq d}c_{ij}x^iy^j\right\}.
  \end{displaymath}
\end{nota}
\begin{defn}
  给定整数$d>0$,和三角形$T$相关的$d$阶 Bernstein 基多项式定义为
  \begin{displaymath}
    B^d_{ijk}:=\frac{d!}{i!j!k!}b^i_1b_2^jb_3^k,\quad i+j+k=d,
  \end{displaymath}
  $i,j,k$是非负整数。
\end{defn}
\begin{prop}
  Bernstein基多项式性质:
  \begin{enumerate}
  \item $ 0\leq B_{ijk}^d(x,y)\leq 1 \quad (x,y) \in T,$
  \item $ \sum_{i+j+k=d}B_{ijk}^d(x,y) \equiv 1\quad (x,y)\in T.$
  \end{enumerate}
\end{prop}


\begin{lem}
  多项式 $\left\{ B^d_{ijk}\right\}$线性独立,
  且构成$d$次多项式${\cal P}_d$线性空间的一组基,维数是
  $\begin{pmatrix}
    d+2\\ 2
  \end{pmatrix}$。
\end{lem}

% \begin{nota}
%   ${\cal S}^0_d(\triangle):=\{s\in C^0(\Omega):s|_{T_i}\in{\cal P}_d ,
%   \quad i=1,...,n_t\}.$
% \end{nota}
% \begin{prop}  \begin{displaymath}
% \textmd{dim} {\cal S}^0_d(\triangle)=n_v+(d-1)n_e+\frac{(d-1)(d-2)}{2}n_t,
% \end{displaymath}
% 其中,$n_v,n_e$和$n_t$分别是三角剖分的点、边和三角形个数。
% \end{prop}

\begin{rem}
  为了达到四阶精度,我们用$p\in {\cal P}_3$
  \begin{displaymath}
    p(x,y)=\sum_{i+j+k=3}c_{ijk}B_{ijk}^d=\sum_{i=1}^{10}c_iB_i,
  \end{displaymath}
  其中$c_i,1\leq i\leq 10$分别
  是
  $c_{300},c_{210},c_{201},c_{120},c_{111},c_{102},c_{030},c_{021},c_{012}$,
  $c_{003}$.
\end{rem}



\begin{rem}
  考虑以下两种局部曲面表示
  \begin{enumerate}
  \item 基于适定结点组生成算法的局部二元三次多项式插值\cite{ShaoThesis};
  \item 基于三角剖分的局部部分插值部分最小二乘拟合
    \cite{schumaker2015spline}.
  \end{enumerate}
\end{rem}

\subsection{基于三角剖分的局部部分插值部分最小二乘拟合}
\label{sec:局部三次样条曲面}
\begin{defn}[算法]\cite{schumaker2015spline}
  对于任意的三角剖分中的三角形,
\begin{enumerate}
\item 以三角形为中心,寻找合适的拟合点。
  取三个顶点star中的所有点。
  如果点大于20,取距离三角形中心最近的20个点。
\item 根据在三个三角形顶点插值,最小二乘拟合其他点的条件,
  计算三次样条基函数的系数。
\end{enumerate}
\end{defn}

\begin{lem}
  设三角形$T$ 的三个顶点分别是$v_1,v_2$ 和 $v_3$.
 给定点列 $\{p:=(x_i,y_i)^m_{i=1}\}$对应的值 $\{z_i\}^m_{i=1}$
 令
  \begin{displaymath}
    O:=[B_1(p_j),...,B_{10}(p_j)]_{j=1}^m,
  \end{displaymath}
  其中$B_1,...,B_{10}$是关于$T$的三次Bernstein基多项式。
  令$\tilde{O}$是$O$去掉第1,7,10列的子矩阵,且满秩。
  则对任意的$\{\omega_i\}_{i=1}^3$,存在一个唯一的二元三次多项式使得
  \begin{displaymath}
    g:=\sum_{i=1}^{10}c_iB_i
  \end{displaymath}
  满足$g(v_i)=w_i,i=1,2,3$,且极小化
  \begin{displaymath}
    \sum_{j=1}^m[g(p_j)-z_j]^2.
  \end{displaymath}
\end{lem}

\begin{pro}
  由Bernstein基多项式的性质可以通过令$c_1 = w_1, c_7 = w_2, c_{10} = w_3$确保$g(v_i)=w_i,i=1,2,3$。
  \begin{displaymath}
    \boldsymbol{c} = \left[
      \begin{matrix}
        c_2\\
        \vdots\\
        c_6\\
        c_8\\
        c_9
      \end{matrix}
    \right], 
    \boldsymbol{b} = \left[
      \begin{matrix}
        z_1\\
        \vdots\\
        z_m
      \end{matrix}
    \right] - w_1O_1-w_2O_7-w_3O_{10}.      
  \end{displaymath}
  为了得到其余的多项式,求解超定方程组
  \begin{displaymath}
    \tilde{O}\boldsymbol{c} = \boldsymbol{b}
  \end{displaymath}
  由于$\tilde{O}$列满秩,等价于求解正规方程
  \begin{displaymath}
    \tilde{O}^T\tilde{O}\boldsymbol{c} = \tilde{O}^T\boldsymbol{b}.
  \end{displaymath}
  得到唯一的最小二乘解。
\end{pro}

\subsection{数值测试结果}

\begin{figure}[H]
  \centering
  \includegraphics[width=0.35\linewidth]{images/tri}
  \caption{加密网格的方法}
\end{figure}
\begin{table}[htbp]
  \begin{center}  
    \begin{tabular}{|l|l|l|l|l|l|l|l|}  
      \hline  
      norm & $|E_h|$ & rate & $|E_{h/2}|$ & rate& $|E_{h/4}|$ &rate &$|E_{h/8}|$ \\
      \hline            
      1-norm & 2.77e-3 & 3.09 & 3.25e-4 & 4.34& 1.60e-5 & 3.89&1.07e-6 \\
      \hline
      2-norm & 5.97e-4 & 3.56 & 5.05e-5 & 4.00& 3.15e-6 & 4.82&1.11e-7 \\
      % 5 & 2.11e-3 & 8.12 & 3.28e-3 & -& 9.88e-06 \\  \hline
      \hline  
    \end{tabular}  
  \end{center}  
  \caption{局部完全插值}  
\end{table}
\begin{table}[h]
  \centering
  \begin{tabular}{|c|c|c|c|c|c|c|}
    \hline
    $|E_h|$ & rate & $|E_{h/2}|$ & rate& $|E_{h/4}|$&rate&$|E_{h/8}|$\\ 
    \hline
    1.74e-5& 5.19& 4.77e-7& 5.10& 1.38e-8& 4.03& 8.48e-10\\ 
    \hline
  \end{tabular}
  \caption{局部部分插值结合最小二乘}
\end{table}



\input{sec/ref}
%\end{CJK*}
\end{document}

%%% Local Variables:
%%% mode: latex
%%% TeX-master: t
%%% End:
